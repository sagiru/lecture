%
% Template to use for seminar, lecture, course or workshops. Mainly for
% students. It is inspired by http://www.minet.uni-jena.de. The base
% configuration is set to create german single side contents. 
%
% Author: Sascha Girrulat <sascha@girrulat.de>
%
\documentclass[
    11pt,
    DIV10,
    ngerman,
    a4paper,
    oneside,
    titlepage,
    parskip=half,
    headings=normal,
    listof=totoc,
    bibliography=totoc,
    index=totoc,
    captions=tableheading,
    final,
]{scrreprt}

\usepackage[ngerman]{babel}
\usepackage[utf8]{inputenc}

\usepackage[T1]{fontenc}
\usepackage{textcomp}
\usepackage{lmodern}
\usepackage{relsize}

\usepackage[dvips,final]{graphicx}
\usepackage{amsmath,amsfonts}

\usepackage{setspace}
\usepackage{geometry}

\usepackage{natbib}

\usepackage[
    bookmarks,
    bookmarksopen=true,
    colorlinks=true,
    linkcolor=black,
    anchorcolor=black,
    citecolor=black,
    filecolor=black,
    menucolor=black,
    urlcolor=blue,
    plainpages=false,
    pdfpagelabels,
    hypertexnames=false,
    linktocpage,
    pdfusetitle 
]{hyperref}

\usepackage{url}
\usepackage{listings}
\usepackage{xcolor} 
\usepackage{chngcntr}
\usepackage{ifthen}
\usepackage{xspace}
\usepackage{fancyhdr}
\usepackage{titlesec}
\usepackage[acronyms, toc, automake, nonumberlist]{glossaries}
\usepackage[figure,table,lstlisting]{totalcount}

\usepackage{booktabs}
\usepackage{todonotes}

\usepackage[latin1]{inputenc}

\newcommand{\titel}{Git}
\newcommand{\tagline}{}
\newcommand{\type}{Seminararbeit} 
\newcommand{\studypath}{Bachelor Informatik}
\newcommand{\autor}{Chuck Norris}
\newcommand{\email}{foo@bar.de}
\newcommand{\matnr}{42}
\newcommand{\institute}{Informatik}
\newcommand{\workgroup}{LG Programmiersysteme}
\newcommand{\university}{FernUniversit�t Hagen \textbullet  Universit�tsstra�e 11 (IZ) \textbullet  58097 Hagen
}
\newcommand{\address}{\workgroup  \textbullet  \institute \\ \university}
\newcommand{\version}{Version 0.1}

\newboolean{final}
\setboolean{final}{false}

\newboolean{appendix}
\setboolean{appendix}{false}

\newboolean{abstract}
\setboolean{abstract}{true}

\newboolean{glossary}
\setboolean{glossary}{true}

\newboolean{preamble}
\setboolean{preamble}{true}

\newboolean{result}
\setboolean{result}{true}

\newboolean{listoffigures}
\setboolean{listoffigures}{false}

\newboolean{listoftables}
\setboolean{listoftables}{false}

\newboolean{listoflistings}
\setboolean{listoflistings}{false}


\makeindex
\makeglossaries
\glstoctrue

\newcommand{\lineDistanceMain}{1.2}
\newcommand{\lineDistanceAppendix}{1.2}

\setlength{\topskip}{\ht\strutbox}
\geometry{paper=a4paper,left=3.5cm, right=2.5cm, top=2.5cm, headsep=1cm}

\titleformat{\chapter}[hang]{\LARGE\bfseries}{\thechapter\quad}{0pt}{}
\titleformat{\section}[hang]{\Large\bfseries}{\thesection\quad}{0pt}{}
\titleformat{\subsection}[hang]{\large\bfseries}{\thesubsection\quad}{0pt}{}
\titleformat{\subsubsection}[hang]{\large\mdseries}{\thesubsubsection\quad}{0pt}{}

\titlespacing{\chapter}{0pt}{-3em}{0pt}
\titlespacing{\section}{0pt}{0pt}{0pt}
\titlespacing{\subsection}{0pt}{0pt}{0pt}
\titlespacing{\subsubsection}{0pt}{0pt}{0pt}
\titlespacing{\paragraph}{0pt}{0pt}{0pt}

% style of header/footer twoside/oneside
\if@twoside
\fancypagestyle{mypagestyle}{%
  \fancyhf{}
  \fancyhead[OR]{\leftmark}
  \fancyfoot[OR]{\thepage}

  \ifthenelse{\boolean{final}}{}{\fancyfoot[OL]{\version}}
  \ifthenelse{\boolean{final}}{}{\fancyhead[OL]{Version vom \today}}

  \fancyhead[EL]{\titel}
  \fancyfoot[EL]{\thepage}

  \ifthenelse{\boolean{final}}{}{\fancyhead[ER]{Version vom \today}}
  \ifthenelse{\boolean{final}}{}{\fancyfoot[ER]{\version}}
}

\else
\fancypagestyle{mypagestyle}{%
  \fancyhf{}
  \fancyhead[R]{\leftmark}
  \fancyfoot[R]{\thepage}

  \ifthenelse{\boolean{final}}{}{\fancyfoot[L]{\version}}
  \ifthenelse{\boolean{final}}{}{\fancyhead[L]{Version vom \today}}

  \fancyhead[R]{\titel}
  \fancyfoot[R]{\thepage}

  \ifthenelse{\boolean{final}}{}{\fancyhead[L]{Version vom \today}}
  \ifthenelse{\boolean{final}}{}{\fancyfoot[L]{\version}}
}

\fi
\pagestyle{mypagestyle}

\renewcommand*{\chapterpagestyle}{mypagestyle} 
\renewcommand{\chaptermark}[1]{\markboth{#1}{}}
\renewcommand{\headfont}{\normalfont}
\renewcommand{\headrulewidth}{0.4pt}
\renewcommand{\footrulewidth}{0.4pt}

\frenchspacing 

% Schusterjungen und Hurenkinder vermeiden
\clubpenalty = 10000
\widowpenalty = 10000 
\displaywidowpenalty = 10000

\lstset{numbers=left, numberstyle=\tiny, numbersep=5pt, breaklines=true}
\lstset{emph={square}, emphstyle=\color{red}, emph={[2]root,base}, emphstyle={[2]\color{blue}}}

\counterwithout{footnote}{chapter}

\definecolor{hellgelb}{rgb}{1,1,0.9}
\definecolor{colKeys}{rgb}{0,0,1}
\definecolor{colIdentifier}{rgb}{0,0,0}
\definecolor{colComments}{rgb}{1,0,0}
\definecolor{colString}{rgb}{0,0.5,0}
\lstset{
    float=hbp,
    columns=flexible,
    tabsize=2,
    frame=L,
    extendedchars=true,
    showspaces=false,
    showstringspaces=false,
    numbers=none,
    basicstyle=\footnotesize,
    breaklines=true,
    breakautoindent=true,
    xleftmargin=0.6cm,
    xrightmargin=0.1cm,
    captionpos=b
}


% define own commands here
%\newcommand{\mycommand}{oldcommand}

\newcommand{\apx}{Anhang}
\newcommand{\gloss}{Glossar}
\newcommand{\abstrc}{Kurzfassung}
\newcommand{\result}{Fazit}
\newcommand{\lookout}{Ausblick}
\newcommand{\target}{Ziel der Arbeit}
\newcommand{\structure}{Aufbau der Arbeit}
\newcommand{\preamble}{Einleitung}


\begin{document}

\title{\titel}
\author{\autor}

\setcounter{secnumdepth}{3}
\setcounter{tocdepth}{3}

\fancyfoot[OR]{}%
\fancyfoot[EL]{}%
\fancyfoot[OL]{}%
\fancyfoot[ER]{}%

\begin{titlepage}
\newgeometry{top=2cm,bottom=2cm,left=2cm,right=2cm}
\begin{figure}
    \begin{minipage}{0.2\textwidth}
        \begin{flushleft}    
            \includegraphics[scale=0.20]{images/logo_left.png}
        \end{flushleft}
    \end{minipage}  
    \begin{minipage}{0.55\textwidth}
        \centering
        \hspace{0.25cm}
    \end{minipage}
    \begin{minipage}{0.2\textwidth}
        \begin{flushleft}   
            \includegraphics[scale=0.5]{images/logo_right.png}
        \end{flushleft}
    \end{minipage} 
\end{figure}
\begin{center}
    \vspace*{0cm}
    \huge{\textbf{\type~im Studiengang \studypath}}
\end{center}
\begin{center}
    \vspace{1cm}
    \Huge{\textbf{\titel}}
    \vspace{1cm}
\end{center}
\ifthenelse{\boolean{final}}{}{
    \begin{center}
        \version ~vom \today        
    \end{center}
}
\begin{center}
    \vspace{1cm}
    \huge{\textbf{\autor\\}}
    \vspace{1cm}
    \LARGE{\matnr\\}
    \vspace{0,5cm}
    \LARGE{\email}
\end{center}
\vspace{1cm}
\begin{center}
    \centering
    \LARGE{\textbf{Betreuer:}} \\
\end{center}
\vspace{0.5cm}
\begin{flushright}
\end{flushright}
\begin{center}
    \small{\workgroup  \textbullet  \institute \\ \university }
\end{center}
\end{titlepage}
\restoregeometry

\ifthenelse{\boolean{final}}{\cleardoublepage}{}


\newgeometry{left=3.5cm, right=2.5cm, top=2.9cm, bottom=2.9cm}

\ifthenelse{\boolean{abstract}}{
  \section*{\abstrc}
\label{sec:abstract}
foobar foobar

  \ifthenelse{\boolean{final}}{\cleardoublepage}{}
}{}

\fancyfoot[OR]{\pagemark}%
\fancyfoot[EL]{\pagemark}%

\tableofcontents

\ifthenelse{\boolean{listoffigures}}{\listoffigures}{}
\ifthenelse{\boolean{listoftables}}{\listoftables}{}

\renewcommand{\lstlistlistingname}{Listings}
\iftotallstlistings\lstlistoflistings\fi

\nomenclature{API}{Application Programming Interface}
\nomenclature{XML}{Extensible Markup Language}


% Topline inside the page header 
\clearpage\markboth{\nomname}{\nomname}

\printnomenclature
\label{cha:abbreviations}
\clearpage

\begin{spacing}{\lineDistanceMain}
\section{\preamble}
\label{cha:preamble}
foobar foobar foobar foobar
\subsection{Motivation}
\label{sec:motivation}
foobar foobar foobar foobar
\subsection{\target}
\label{sec:target}
foobar foobar foobar foobar
\subsection{\structure}
\label{sec:structure}
foobar foobar foobar foobar foobar foobar foobar foobar
foobar foobar foobar foobar foobar foobar foobar foobar
foobar foobar foobar foobar foobar foobar foobar foobar

\section{Kapitel1}
\label{cha:Kapitel1}
foobar foobar foobar foobar foobar foobar foobar foobar
\subsection{Section1}
\label{cha:Section1}
foobar foobar foobar foobar foobar foobar foobar foobar
\section{Kapitel2}
foobar foobar foobar foobar foobar foobar foobar foobar

\section{\result}
\label{cha:result}
foobar foobar foobar foobar 
\section{\lookout}
\label{cha:lookout}
foobar foobar foobar foobar 


\end{spacing}

% appendix if selected
\ifthenelse{\boolean{appendix}}{
  \begin{spacing}{\lineDistanceAppendix}
  \begin{appendix}
      \chapter{\apx}
      \label{sec:appendix}
      \input{appendix}
  \end{appendix}
  \end{spacing}
}{}

% glossary if selected
\ifthenelse{\boolean{glossary}}{
\newglossaryentry{computer}{
  name=Computer,
  description={ist ein elektronisches Ger\"at, das Daten verarbeitet}
}

\label{sec:glossary}

% \glsaddall

\printglossaries
\addcontentsline{toc}{chapter}{Glossar}
}{}

\clearpage
\bibliography{bibliography}
\bibliographystyle{alphadin}

\end{document}
